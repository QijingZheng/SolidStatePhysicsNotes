\begin{frame}
  \frametitle{Phonon Polarization Vectors}

  The eigenvectors $\boldsymbol{\eta}^\sigma(\boldsymbol{q})$ of the Hermitian
  matrix $D_{s\alpha,t\beta}(\boldsymbol{q})$ are called the phonon polarization
  vector.

  \begin{itemize}
    \setlength\itemsep{\smallskipamount}
    
    \item The polarization vector is \emph{cell-periodic}.
    \begin{equation*}
        {u}_{s\alpha}^j(t) =
        {{\eta}^\sigma_{s\alpha}(\boldsymbol{q}) \over \sqrt{M_s}}\,
        e^{i\boldsymbol{q}\mathcolor{red}{\boldsymbol{x}_j}} \,e^{- i\omega_\sigma t}
    \end{equation*}
    So the solution is a cell-periodic part multiply by
    $e^{i\boldsymbol{q}\boldsymbol{x}}$ --- Bloch's theorem.

    \item Orthogonalization relation:
    \begin{equation*}
      \sum_{s\alpha}
      \eta^{\sigma'}_{s\alpha}(\boldsymbol{q})\,
      \eta^{\sigma}_{s\alpha}(\boldsymbol{q})
      = \delta_{\sigma\sigma'};
      \qquad
      \sum_{\sigma}
      \eta^{\sigma}_{s\alpha}(\boldsymbol{q})\,
      \eta^{\sigma}_{t\beta}(\boldsymbol{q})
      = \delta_{st} \, \delta_{\alpha\beta}
    \end{equation*}
    \item Relation to phonon displacemoment --- direction and amplitude of the vibration.
      \begin{equation*}
      \xi^{\sigma}_{s\alpha} = {1\over\sqrt{M_s}}\,\eta^{\sigma}_{s\alpha}(\boldsymbol{q})
      \end{equation*}
      
    \item At some high-symmetry $\boldsymbol{q}$-path
      \begin{equation*}
        \begin{cases}
          \boldsymbol{q} \parallel \boldsymbol{\eta}(\boldsymbol{q}) & \text{Longitudinal Wave} \cr
          \boldsymbol{q} \perp     \boldsymbol{\eta}(\boldsymbol{q}) & \text{Transverse Wave} \cr
        \end{cases}
      \end{equation*}
  \end{itemize}
\end{frame}

%%% Local Variables:
%%% mode: latex
%%% TeX-master: t
%%% End:
