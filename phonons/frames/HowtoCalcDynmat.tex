\begin{frame}
  \frametitle{How to Calculate the Dynamical Matrix}
  The definition of the dynamical matrix
  \begin{equation*}
    \mathcolor{black}{
      D_{s\alpha,t\beta}(\boldsymbol{q})
    }
    =
    {1\over\sqrt{M_sM_t}\,}
    \sum_{l=\text{-}\infty}^{\infty}
    C_{s\alpha,t\beta}(0, l)
    e^{i\boldsymbol{q}\boldsymbol{x}_l}
    \approx
    {1\over\sqrt{M_sM_t}\,}
    \sum_{|l| < l_{\text{cut}}}
    C_{s\alpha,t\beta}(0, l)
    e^{i\boldsymbol{q}\boldsymbol{x}_l}
  \end{equation*}

  \begin{itemize}
  
  % \item Knowing the IFC $C_{s\alpha,t\beta}(0, l)$, we can get the dynamical matrix at
  % arbitrary $\boldsymbol{q}$.

  \item Frozen phonon method with a supercell
    
    \begin{center}
        \begin{tikzpicture}
        [scale=1.0]
        \foreach \i in {0, 1,...,6}{
            \foreach \j in {0, 1}{
            \foreach \x/\y/\c in {0.3/0.3/blue, 0.7/0.7/blue, 0.2/0.8/red, 0.8/0.2/red}{
                \shade[ball color=\c] (\x + \i, \y + \j) circle (0.08);
            }
            }
        }
        \draw[help lines, thin, draw=gray] (0, 0) grid (7, 2);
        \draw[help lines, thick, draw=red] (0, 0) grid (1, 1);


        % move one of the atoms
        \draw[line width=2.0pt, draw=black, densely dotted] (0.6, 0.4) circle (0.08);
        \shade[ball color=blue!40] (0.6, 0.4) circle (0.08);
        \draw[solid, draw=YellowGreen, line width=1.0pt, ->, >=stealth] (0.3, 0.3) -- (0.6, 0.4);

        % add the text
        \node[
        draw=red, rounded corners=3pt,
        anchor=north west, align=center
        ] (move_the_atom_text)
        at (0.1, -0.1) {
          \small
          move atoms in this cell % by a small displacement
        };
        \draw[
        draw=black, line width=0.5pt, ->, >=stealth,
        out=120, in=-150
        ]
        (move_the_atom_text.north west) to (0.2, 0.3);

        \node[
        draw=blue, rounded corners=3pt,
        anchor=south east, align=center
        ] (measure_the_force_text)
        at (7.0, 2.1) {
          \small
          measure the force of this atom
        };
        \draw[
        draw=blue, line width=0.5pt, ->, >=stealth,
        out=-90, in=120
        ]
        (measure_the_force_text.south) to (6.2, 1.8);

        \node[
        draw=black, rounded corners=3pt,
        anchor=east, text width=2.0cm, align=center
        ] (real_ifc)
        at (-0.5, 1.0) {
          \small
          IFC in real space:
          \begin{equation*}
            % C_{s\alpha,t\beta}(0, l) =
            \frac{
            \partial^2 E^0_{\text{tot}}
            }{
            \partial {u}_{s\alpha}^0
            \partial {u}_{t\beta}^l
            }
          \end{equation*}
        };
        \end{tikzpicture}
    \end{center}

    \medskip
    or
    \medskip
    \begin{center}
        \begin{tikzpicture}
        [scale=1.0]
        \foreach \i in {0, 1,...,6}{
            \foreach \j in {0}{
            \foreach \x/\y/\c in {0.3/0.3/blue, 0.7/0.7/blue, 0.2/0.8/red, 0.8/0.2/red}{
                \shade[ball color=\c] (\x + \i, \y + \j) circle (0.08);
            }
            }
        }
        \draw[help lines, thin, draw=gray] (0, 0) grid (7, 1);
        % \draw[thin, magenta] (0.0, 0.6) -- (7.0, 0.6);
        \draw[line width=0.8pt, color=olive, domain=0:7.0, densely dotted] plot[samples=300]
        (
        \x, {0.3 + 0.3 * cos((\x - 0.3) * 2. * pi / 7 r)}
        );
        \foreach \x in {0,1,...,6}{
          \pgfmathparse{0.3 * cos((\x) * 2 * pi / 7. r)}
          \let \y = \pgfmathresult
          \shade[] (\x +0.3, 0.3 + \y) circle (0.05);
          \draw[->, >=stealth, thick, magenta] (\x+0.3, 0.3) -- ++(0.0, \y);
        };

        \node[
        draw=black, rounded corners=3pt,
        anchor=east, text width=2.0cm, align=center
        ] (real_ifc)
        at (-0.5, 0.5) {
          \small
          Dynamical matrix:
          \begin{equation*}
            D_{s\alpha,t\beta}(\boldsymbol{q})
          \end{equation*}
        };
        \end{tikzpicture}
    \end{center}
  \end{itemize}
\end{frame}