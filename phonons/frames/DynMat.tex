\begin{frame}
  \frametitle{The Dynamical Matrix}
  The Dynamical Matrix
  \begin{equation*}
    \mathcolor{black}{
      D_{s\alpha,t\beta}(\boldsymbol{q})
    }
    % =
    % {1\over\sqrt{M_sM_t}\,}
    % \mathcolor{black}{\sum_k}
    % \mathcolor{black}{
    %   \frac{
    %   \partial^2 E^0_{\text{tot}}
    %   }{
    %   \partial {u}_{s\alpha}^j
    %   \partial {u}_{t\beta}^k
    %   }
    %   e^{i\boldsymbol{q}(\boldsymbol{x}_k - \boldsymbol{x}_j)}
    % }
    =
    {1\over\sqrt{M_sM_t}\,}
    \mathcolor{black}{\sum_l}
    \mathcolor{black}{
      \frac{
      \partial^2 E^0_{\text{tot}}
      }{
      \partial {u}_{s\alpha}^0
      \partial {u}_{t\beta}^l
      }
      e^{i\boldsymbol{q}\boldsymbol{x}_l}
    }
    =
    {1\over\sqrt{M_sM_t}\,}
    \sum_l
    C_{s\alpha,t\beta}^{0, l}\,
    e^{i\boldsymbol{q}\boldsymbol{x}_l}
    % &=
    % {1\over\sqrt{M_sM_t}\,}
    % \sum_k
    % C_{s\alpha,t\beta}^{j, k}\,
    % e^{i\boldsymbol{q}(\boldsymbol{x}_k - \boldsymbol{x}_j)} \cr
    % &=
    % {1\over\sqrt{M_sM_t}\,}
    % \sum_l
    % C_{s\alpha,t\beta}^{0, l}\,
    % e^{i\boldsymbol{q}\boldsymbol{x}_l}
  \end{equation*}

  \begin{itemize}
  \item
    % $D_{s\alpha,t\beta}(\boldsymbol{q})$ can be seen as the second
    % derivative of $E^0_{\text{tot}}$ with respect to the amplitude of a lattice
    % distortion of definite wave vector $\boldsymbol{q}$
    If we define the distortion pattern $\boldsymbol{u}^l_{s}(\boldsymbol{q}) =
    \boldsymbol{v}_{s}(\boldsymbol{q})\, e^{i\boldsymbol{q}\boldsymbol{x}_l}$
    \begin{equation*}
      D_{s\alpha,t\beta}(\boldsymbol{q})
      =
      {1 \over N}
      {1\over\sqrt{M_sM_t}\,}
      \frac{
      \partial^2 E^0_{\text{tot}}
      }{
      \partial {v}^*_{s\alpha}(\boldsymbol{q})
      \partial {v}_{t\beta}(\boldsymbol{q})
      }
    \end{equation*}
      
  \item Dynamical matrix is Hermitian and admit real eigenvalues $\omega^2(\boldsymbol{q})$
    \begin{equation*}
      D_{s\alpha,t\beta}(\boldsymbol{q})
      =
      D^*_{t\beta,s\alpha}(\boldsymbol{q})
    \end{equation*}
    
    % \begin{block}{\textsc{Proof}}
    \textcolor{blue}{\textsc{Proof}}
      \small
      \begin{columns}[t]
        \begin{column}{0.5\textwidth}
          \begin{align*}
            D_{s\alpha,t\beta}(\boldsymbol{q})
            &=
            {1\over\sqrt{M_sM_t}\,}
            \sum_l
            C_{s\alpha,t\beta}^{0, l}\,
              e^{i\boldsymbol{q}\boldsymbol{x}_l} \cr
            &= 
            {1\over\sqrt{M_sM_t}\,}
            \sum_l
            C_{s\alpha,t\beta}^{\text{-}l, 0}\,
              e^{i\boldsymbol{q}\boldsymbol{x}_l} \cr
            &= 
            {1\over\sqrt{M_sM_t}\,}
            \sum_l
            C_{t\beta,s\alpha}^{0, \text{-}l}\,
              e^{i\boldsymbol{q}\boldsymbol{x}_l}
            \tikzmark{proof_1}
          \end{align*}
        \end{column}
        \begin{column}{0.4\textwidth}
          \begin{align*}
            \tikzmark{proof_2}
            D^*_{s\alpha,t\beta}(\boldsymbol{q})
            &=
            {1\over\sqrt{M_sM_t}\,}
            \sum_l
            C_{t\beta,s\alpha}^{0, \text{-}l}\,
              e^{\text{-}i\boldsymbol{q}\boldsymbol{x}_l} \cr
            &=
            {1\over\sqrt{M_sM_t}\,}
            \sum_l
            C_{t\beta,s\alpha}^{0, l}\,
              e^{i\boldsymbol{q}\boldsymbol{x}_l} \cr
            &= 
            D_{t\beta,s\alpha}(\boldsymbol{q})
          \end{align*}
        \end{column}
      \end{columns}

      \begin{tikzpicture}
        [overlay, remember picture]
        \draw[
            ->, >=stealth, line width=0.4pt,
            out=0, in=-180
        ]
        (proof_1.east) to (proof_2.west);
      \end{tikzpicture}
    % \end{block}
  \end{itemize}
\end{frame}

%%% Local Variables:
%%% mode: latex
%%% TeX-master: t
%%% End:
