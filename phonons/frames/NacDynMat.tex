\begin{frame}
  \frametitle{Long-wavelength limit in polar materials}
  The dynamical matrix in polar materials can be separated into two parts
  \footnote{
    \href{https://journals.aps.org/prb/abstract/10.1103/PhysRevB.55.10355}
         {X. Gonze and C. Lee, \textit{Phys. Rev. B}\/ 55, 10355(1997).}
  }
  \begin{equation*}
    D_{s\alpha,t\beta}(\boldsymbol{q} \to 0) = 
    D^{\text{an}}_{s\alpha,t\beta}(\boldsymbol{q} = 0) +
    D^{\text{na}}_{s\alpha,t\beta}(\boldsymbol{q} \to 0)
  \end{equation*}
  where the nonanalytic part is written as
  \begin{equation}
    \label{eq:dynmat_nac}
    D^{\text{na}}_{s\alpha,t\beta}(\boldsymbol{q} \to 0)
    =
    {1\over\sqrt{M_sM_t}\,}
    {4\pi e^2\over\Omega}
    \frac{
      \left(
        \sum_{\gamma} q_\gamma Z^{*\gamma\alpha}_s
      \right) 
      \left(
      \sum_{\mu} q_\mu Z^{*\mu\beta}_t
      \right) 
    }{
      \sum_{\gamma\mu} q_\gamma \, \epsilon_\infty^{\gamma\mu} \,q_\mu 
    }
  \end{equation}
  \begin{itemize}
    \item
      $Z^*_s$ is the Born effective charge tensor for atom $s$
        \begin{equation*}
        Z_s^{*\alpha\beta} = \Omega_0
        \frac{
            \partial {\cal P}_\beta
        }{
            \partial u_{s\alpha}
        }
        =
        \frac{
            \partial F_{s\alpha}
        }{
            \partial {\cal E}_\beta
        }
        \end{equation*}

        \begin{enumerate}
        \setlength\itemsep{\smallskipamount}
        \setbeamercolor{local structure}{fg=red}
        \item The response of the polarization per unit cell along the direction
          $\beta$ induced by a displacement along the direction $\alpha$ of the atoms
          belonging to the sublattice $s$, under the condition of a zero electric
          field.
          
        \item The force on the atom $s$ along $\alpha$ induced by the macroscopic
          field along $\beta$.
        \end{enumerate}
        
    \item $\epsilon_\infty$ is the \emph{electronic} dielectric tensor of the
      crystal, i.e. the static dielectric constant with clamped nuclei.
  \end{itemize}
\end{frame}

%%% Local Variables:
%%% mode: latex
%%% TeX-master: t
%%% End:
