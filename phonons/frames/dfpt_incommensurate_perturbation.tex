\begin{frame}[allowframebreaks]
  \frametitle{DFPT --- Incommensurate Perturbations}

  Add a monochromatic perturbation characterized by generic wavevector $\bdm{q}$ to the periodic potential

  \begin{equation*}
    % \frac{
    %   \partial V_{ext}^{\bdm{q}}(\boldr + \bdm{x})
    % }{
    %   \partial \mu
    % }
    % =
    % e^{i\bdm{q}\bdm{x}}\,
    % \frac{
    %   \partial V_{ext}^{\bdm{q}}(\boldr)
    % }{
    %   \partial \mu
    % }
    V_{ext}(\boldr + \bdm{x}_j)
    =
    V_{ext}(\boldr)
    ;
    \qquad
    \Delta V_{ext}^{\bdm{q}}(\boldr + \bdm{x}_j)
    =
    e^{i\bdm{q}\cdot\bdm{x}_j}\,
    \Delta V_{ext}^{\bdm{q}}(\boldr)
  \end{equation*}

  It can be shown that all the linear responses are also characterized by wavevector $\bdm{q}$
  \begin{align*}
    \Delta V_{KS}^{\bdm{q}}(\boldr + \bdm{x}_j) &= e^{i\bdm{q}\cdot\bdm{x}_j}\,
                                                  \Delta V_{KS}^{\bdm{q}}(\boldr) \\[3pt]
    \Delta\rho^{\bdm{q}}(\boldr + \bdm{x}_j) &= e^{i\bdm{q}\cdot\bdm{x}_j}\,
                                                \Delta\rho^{\bdm{q}}(\boldr) \\[3pt]
    \Delta \psi_{n,\bdm{k}}^{\bdm{q}}(\boldr + \bdm{x}_j) &=
                                                     e^{i\bdm{q}\cdot\bdm{x}_j}\,
                                                     \Delta \psi_{n,\bdm{k}}^{\bdm{q}}(\boldr)
  \end{align*}
  Let us define the related periodic quantities
  \begin{align*}
    \Delta V_{ext}^{\bdm{q}}(\boldr) &= e^{i\bdm{q}\cdot\boldr}\, {\tilde V}_{ext}^{\bdm{q}}(\boldr);
    &
     {\tilde V}_{ext}^{\bdm{q}}(\boldr + \bdm{x}_j) &= 
     {\tilde V}_{ext}^{\bdm{q}}(\boldr);
    \\[3pt]
    \Delta V_{KS}^{\bdm{q}}(\boldr) &= e^{i\bdm{q}\cdot\boldr}\, {\tilde V}_{KS}^{\bdm{q}}(\boldr);
    &
     {\tilde V}_{KS}^{\bdm{q}}(\boldr + \bdm{x}_j) &= 
     {\tilde V}_{KS}^{\bdm{q}}(\boldr);
    \\[3pt]
    \Delta\rho^{\bdm{q}}(\boldr) &= e^{i\bdm{q}\cdot\boldr}\, {\tilde\rho}^{\bdm{q}}(\boldr);
    &
     {\tilde \rho}^{\bdm{q}}(\boldr + \bdm{x}_j) &= 
     {\tilde \rho}^{\bdm{q}}(\boldr);
    \\[3pt]
    \psi_{n,\bdm{k}}(\boldr) &= e^{i\bdm{k}\cdot\boldr}\, u_{n,\bdm{k}}(\boldr);
    &
     u_{n,\bdm{k}}(\boldr + \bdm{x}_j) &= 
     u_{n,\bdm{k}}(\boldr); \\[3pt]
    \Delta \psi_{n,\bdm{k}}^{\bdm{q}}(\boldr) &= e^{i(\bdm{k} + \bdm{q})\cdot\boldr}\, \Delta u_{n,\bdm{k}}^{\bdm{q}}(\boldr);
    &
     \Delta u_{n,\bdm{k}}^{\bdm{q}}(\boldr + \bdm{x}_j) &= 
     \Delta u_{n,\bdm{k}}^{\bdm{q}}(\boldr);
  \end{align*}

  % Within DFPT, the response to perturbations of different wavelengths are decoupled.
  \break

  Substitute into Eq.~\ref{eq:dfpt_eq_final}
  \begin{equation}
    \label{eq:dfpt_eq_periodic}
    \biggl[
    H_{KS}(\boldr)
    +
    \mathcolor{olive}{
      \alpha P_v
    }
    - \varepsilon_{n,\bdm{k}}
    \biggr]
    \mathcolor{ForestGreen}{
        P_c
        e^{i(\bdm{k}+\bdm{q})\cdot\boldr}\,
        \frac{
        \partial u_{n,\bdm{k}}^{\bdm{q}}(\boldr)
        }{
        \partial\mu
        }
    }
    =
    -P_c
    e^{i(\bdm{k}+\bdm{q})\cdot\boldr}\,
    \mathcolor{blue}{
    \frac{
      \partial {\tilde V}^{\bdm{q}}_{KS}(\boldr)
    }{
      \partial\mu
    }
    }
    u_{n,\bdm{k}}(\boldr)
  \end{equation}
  Apply the phase $e^{-i(\bdm{k}+\bdm{q})\cdot\boldr}$ to the both sides of Eq.~\ref{eq:dfpt_eq_periodic}
   \begin{equation}
    \label{eq:dfpt_eq_periodic_final}
    \biggl[
    % e^{-i(\bdm{k}+\bdm{q})\cdot\boldr}
    H^{\bdm{k}+\bdm{q}}_{KS}(\boldr)
    % e^{i(\bdm{k}+\bdm{q})\cdot\boldr'}
    +
    \mathcolor{olive}{
      \alpha P_v^{\bdm{k}+\bdm{q}}
    }
    - \varepsilon_{n,\bdm{k}}
    \biggr]
    \mathcolor{ForestGreen}{
        P_c^{\bdm{k}+\bdm{q}}
        \frac{
        \partial u_{n,\bdm{k}}^{\bdm{q}}(\boldr)
        }{
        \partial\mu
        }
    }
    =
    -P_c^{\bdm{k}+\bdm{q}}
    \mathcolor{blue}{
    \frac{
      \partial {\tilde V}^{\bdm{q}}_{KS}(\boldr)
    }{
      \partial\mu
    }
    }
    u_{n,\bdm{k}}(\boldr)
  \end{equation}

  where 
  \begin{gather*}
    \mathcolor{blue}{
    \frac{
      \partial {\tilde V}_{KS}(\boldr)
    }{
      \partial \mu
    }}
    =
    \frac{
      \partial {\tilde V}_{ext}(\boldr)
    }{
      \partial \mu
    }
    +
    \int\, \frac{
      1
    }{
      |\boldr - \boldr'|
    }
    \mathcolor{red}{
      \frac{
      \partial {\tilde\rho}^{\bdm{q}}(\boldr')
      }{
      \partial \mu
    }} \, \mathrm{d}\boldr'
    +
    \int\,
    \frac{
      \mathrm{d} V_{xc}
    }{
      \mathrm{d} \rho(\boldr')
    }
    \mathcolor{red}{
      \frac{
      \partial {\tilde\rho}^{\bdm{q}}(\boldr')
      }{
      \partial \mu
    }} \, \mathrm{d}\boldr'
    \\[3pt]
    \mathcolor{red}{
        \frac{
        \partial {\tilde\rho}^{\bdm{q}}(\boldr')
        }{
        \partial \mu
        }
    }
    =
    2\sum_{n,\bdm{k}}^{occ}
      u_{n,\bdm{k}}^*(\boldr)
      \mathcolor{ForestGreen}{
        P_c^{\bdm{k} + \bdm{q}}
        \frac{
            \partial u_{n,\bdm{k}}^{\bdm{q}}(\boldr)
        }{
            \partial \mu
        }
      } 
    \\[3pt]
    P_v^{\bdm{k} + \bdm{q}}
    =
    \sum_n^{occ} \ket{u_{n,\bdm{k}+\bdm{q}}}\bra{u_{n,\bdm{k}+\bdm{q}}}
    ;
    \quad
    P_c^{\bdm{k} + \bdm{q}}
    =
    \mathbb{1} -  P_v^{\bdm{k} + \bdm{q}}
  \end{gather*}

  The treatment of perturbations incommensurate with the unperturbed system
  periodicity is mapped onto the \emph{original periodic system}.
 
\end{frame}

%%% Local Variables:
%%% mode: latex
%%% TeX-master: t
%%% End:
