\begin{frame}
  \frametitle{LO-TO Splitting}
  % \begin{center}
  %   \resizebox{0.9\textwidth}{!}
  %   {
  %       \begin{tikzpicture}
  %       [scale=1.0]
  %       \foreach \shift in {0, 1}{
  %           \foreach \j in {0, 1, 2}{
  %           \foreach \i in {0, 1, 2,..., 16}{
  %               \pgfmathparse{\i + \j}
  %               \let \n = \pgfmathresult

  %               \ifthenelse{
  %               \isodd{\n}
  %               }{
  %               \pgfmathparse{0.5 * sin(pi * (\i - 1 + mod(\j, 2)) / 4 r)}
  %               \let \A = \pgfmathresult

  %               % plot the atoms
  %               \shade[ball color=blue] (\i, \j + \shift * 4) circle (0.15);
  %               \ifthenelse{\shift=0}{
  %                   % the longitudinal wave for the negative ions
  %                   \draw[-{Stealth[length=6pt, open]}, tips=proper, line width=1pt] (\i, \j + \shift * 4) -- ++ (-\A, 0);
  %               }{
  %                   % the transverse wave for the negative ions
  %                   \draw[-{Stealth[length=6pt, open]}, tips=proper, line width=1pt] (\i, \j + \shift * 4) -- ++ (0, -\A);
  %               }
  %               }{
  %               \pgfmathparse{0.5 * sin(pi / 4 * (\i + mod(\j, 2))  r)}
  %               \let \A = \pgfmathresult

  %               \shade[ball color=red] (\i, \j + \shift * 4) circle (0.15);
  %               \ifthenelse{\shift=0}{
  %                   % the transverse wave for the positive ions
  %                   \draw[-{Stealth[length=6pt, open]}, tips=proper, line width=1pt] (\i, \j + \shift * 4) -- ++ (\A, 0);
  %               }{
  %                   \draw[-{Stealth[length=6pt, open]}, tips=proper, line width=1pt] (\i, \j + \shift * 4) -- ++ (0, \A);
  %               }
  %               }
  %           }
  %           }
  %       }

  %       % Add LO/TO labels
  %       \node[align=center, anchor=east] at (-0.5, 1.0) {\textbf{\large LO Phonon}};
  %       \node[align=center, anchor=east] at (-0.5, 5.0) {\textbf{\large TO Phonon}};
  %       % the q direction
  %       \draw[-{Stealth[length=6pt, open]}, draw, line width=0.8pt]
  %       (0.0, -1) --
  %       (6.0, -1)
  %       node[below=3pt] {\large $\boldsymbol{q}$};
  %       \end{tikzpicture}
  %   }
  % \end{center}
  \begin{center}
    \resizebox{0.80\textwidth}{!}
    {
        \begin{tikzpicture}
        [scale=1.0]
        \foreach \shift in {0, 1}{
            \foreach \j in {0, 1, 2}{
            \foreach \i in {0, 1, 2,..., 8}{
                \pgfmathparse{\i + \j}
                \let \n = \pgfmathresult

                \ifthenelse{
                \isodd{\n}
                }{
                \pgfmathparse{0.5 * sin(pi * (\i - 1 + mod(\j, 2)) / 4 r)}
                \let \A = \pgfmathresult

                % plot the atoms
                \shade[ball color=blue] (\i + \shift * 10, \j) circle (0.15);
                \ifthenelse{\shift=0}{
                    % the longitudinal wave for the negative ions
                    \draw[-{Stealth[length=6pt, open]}, tips=proper, line width=1pt] (\i + \shift * 10, \j) -- ++ (-\A, 0);
                }{
                    % the transverse wave for the negative ions
                    \draw[-{Stealth[length=6pt, open]}, tips=proper, line width=1pt] (\i + \shift * 10, \j) -- ++ (0, -\A);
                }
                }{
                    \pgfmathparse{0.5 * sin(pi / 4 * (\i + mod(\j, 2))  r)}
                    \let \A = \pgfmathresult

                    \shade[ball color=red] (\i + \shift * 10, \j) circle (0.15);
                    \ifthenelse{\shift=0}{
                        % the longitudinal wave for the positive ions
                        \draw[-{Stealth[length=6pt, open]}, tips=proper, line width=1pt] (\i + \shift * 10, \j) -- ++ (\A, 0);
                    }{
                        % the transverse wave for the positive ions
                        \draw[-{Stealth[length=6pt, open]}, tips=proper, line width=1pt] (\i + \shift * 10, \j) -- ++ (0, \A);
                    }
                }

                % the nodal plane
                \pgfmathparse{int(mod(\i, 4))}
                \let \n = \pgfmathresult
                \ifthenelse{\n = 0}{
                    \draw[gray, dashed, ultra thin] (\i + \shift * 10, 0.0) -- ++(0.0, -1);
                }{}
              }
            }
        }

        % Add LO/TO labels
        \node[above=3pt, align=center, anchor=south] at ( 4.0, 3.0)
        {\textbf{\large LO Phonon}};
        \node[above=3pt, align=center, anchor=south] at (14.0, 3.0)
        {\textbf{\large TO Phonon}};
        % the q direction
        \draw[-{Stealth[length=6pt, open]}, draw, line width=0.8pt]
        (7.0, -1.5) --
        (11.0, -1.5)
        node[below=3pt] {\large $\boldsymbol{q}$};
        % \draw[-{Stealth[length=6pt, open]}, draw, line width=0.8pt]
        % (11.0, -1.5) --
        % (17.0, -1.5)
        % node[below=3pt] {\large $\boldsymbol{q}$};

        \foreach \i in {1, 3, 5, ..., 10}{
          \node[align=center, blue] at (0.0, -\i * 1 / 10){\huge\textbf{-}};
          \node[align=center, red] at (4.0, -\i * 1 / 10){\textbf{+}};
          \node[align=center, blue] at (8.0, -\i * 1 / 10){\huge\textbf{-}};
        }
        \draw[-{Stealth[length=6pt, open]}, draw, line width=0.8pt]
        (3.5, -0.5) -- (0.5, -0.5)
        node[below=3pt, pos=0.5] {\large $\boldsymbol{E}$};

        \draw[-{Stealth[length=6pt, open]}, draw, line width=0.8pt]
        (4.5, -0.5) -- (7.5, -0.5)
        node[below=3pt, pos=0.5] {\large $\boldsymbol{E}$};

        \foreach \i in {0, 1, 2, ..., 10}{
          \node[align=center, red ] at (11.0 + \i * 2 / 10,  2.5){\textbf{+}};
          \node[align=center, blue] at (11.0 + \i * 2 / 10, -0.5){\huge\textbf{-}};

          \node[align=center, blue] at (15.0 + \i * 2 / 10,  2.5){\textbf{+}};
          \node[align=center, red ] at (15.0 + \i * 2 / 10, -0.5){\huge\textbf{-}};
        }
        \draw[-{Stealth[length=6pt, open]}, draw, line width=0.8pt]
        (11.5, 1.8) -- (11.5, 0.2)
        node[below=3pt] {\large $\boldsymbol{E}$};

        \draw[-{Stealth[length=6pt, open]}, draw, line width=0.8pt]
        (16.5, 0.2) -- (16.5, 1.8)
        node[above=3pt] {\large $\boldsymbol{E}$};
        \end{tikzpicture}
    }
  \end{center}
\end{frame}

%%% Local Variables:
%%% mode: latex
%%% TeX-master: t
%%% End:
