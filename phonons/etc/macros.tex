\newcommand{\angstrom}{\text{\normalfont\AA}}
\newcommand{\abinitio}[0]{\textit{ab initio}}
\newcommand{\schro}[0]{Schr\"odinger}

\newcommand{\rt}{(\mathbf{r},t)}
\newcommand{\rR}{(\mathbf{r},\mathbf{R})}
\newcommand{\prR}{(\mathbf{r};\mathbf{R})}
\newcommand{\Rt}{(\mathbf{R},t)}
\newcommand{\rRt}{(\mathbf{r},\mathbf{R},t)}
\newcommand{\hamil}[0]{\hat{\cal H}}
\newcommand{\rvec}[0]{\mathbf{r}}
\newcommand{\svec}[0]{\mathbf{s}}
\newcommand{\Rvec}[0]{\mathbf{R}}
\newcommand{\boldr}[0]{\boldsymbol{r}}

\newcommand{\dt}[1]{\mathrm{d}#1}
\newcommand{\onehalf}[0]{\frac{1}{2}}

\newcommand{\BOsup}[0]{\textnormal{\tiny BO}}
% \newcommand{\action}{{\cal A}}

\newcommand{\EYSR}{Elliot-Yafet}
\newcommand{\DPSR}{D'yakonov-Perel'}

\renewcommand{\Re}{\operatorname{Re}}
\renewcommand{\Im}{\operatorname{Im}}
\newcommand{\tr}{\operatorname{tr}}
% \DeclareMathOperator{\Hom}{Hom}

\newcommand{\shortminus}{\text{-}}

\makeatletter
\newcommand*{\rom}[1]{\expandafter\@slowromancap\romannumeral #1@}
\makeatother

\newcommand*{\info}[4][16.3]{%
  \node[
  annotation, #3, scale=0.65, text width = #1em,
  inner sep = 6pt
  ] at (#2) {%
  \list{$\bullet$}{\topsep=0pt\itemsep=0pt\parsep=0pt
    \parskip=0pt\labelwidth=8pt\leftmargin=8pt
    \itemindent=0pt\labelsep=2pt}%
    #4
  \endlist
  };
}

\newcommand{\tikzmark}[1]{\tikz[baseline,remember picture] \coordinate (#1) {};}
\newcommand{\tikzremember}[2]{
  \tikz[baseline,inner sep=0] \node[anchor=base] (#1) {#2};
}

\makeatletter
\renewcommand*\env@matrix[1][\arraystretch]{%
  \edef\arraystretch{#1}%
  \hskip -\arraycolsep
  \let\@ifnextchar\new@ifnextchar
  \array{*\c@MaxMatrixCols c}}
\makeatother

\newcommand*\colvec[3][]{
    \begin{pmatrix}[1.2]\ifx\relax#1\relax\else#1\\\fi#2\\#3\end{pmatrix}
}

\newenvironment {annotatedFigure}[1]{
  \centering
  \begin{tikzpicture}
    \node[anchor=south west,inner sep=0] (image) at (0,0) {#1};
        \begin{scope}[x={(image.south east)},y={(image.north west)}]}
    {\end{scope}
  \end{tikzpicture}}

\newcommand{\zjball}[2][1.5pt]{
    \tikz[baseline=(current bounding box.south)] \shade[
        shading=ball, ball color=#2,
    ] (0.0, 0.0) circle (#1);
}

\makeatletter
\def\mathcolor#1#{\@mathcolor{#1}}
\def\@mathcolor#1#2#3{%
  \protect\leavevmode
  \begingroup
    \color#1{#2}#3%
  \endgroup
}         
\makeatother
